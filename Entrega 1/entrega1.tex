\documentclass[spanish, a4paper, 12pt] {article}
\usepackage[spanish]{babel}
\usepackage[utf8]{inputenc}
\usepackage{amsmath}
\usepackage{amssymb}
\usepackage{amsfonts}
\usepackage{latexsym}
\usepackage{mathtools}
\usepackage{anysize}
%\marginsize{2cm}{2cm}{2cm}{3cm}
\newcommand\eqdef{\stackrel{\mathclap{\mbox{\tiny{def}}}}{=}}
\newcommand\eqac{\stackrel{\mathclap{\mbox{*}}}{=}}

\usepackage{graphicx}
\usepackage{hyperref}
\usepackage{float}
\usepackage{verbatim}
\DeclareGraphicsExtensions{.pdf,.png,.jpg}

\begin{document}
\title{Entrega 1}
\author{Marco Antonio Garrido Rojo\thanks{\url{https://github.com/MaSteve/EDIF} Esta entrega puede encontrarse en GitHub.}}
\date{}
\maketitle
Sea la función $g(x) = 5x^{\frac{4}{5}}$ a lo largo del ejercicio.
\begin{itemize}
\item{
\textbf{Apartado 1} Estudia la continuidad y derivabilidad de la función $g(x), x \in \mathbb{R}$. Determinar si es lipschitziana en algún intervalo $[-r, r], r > 0$.

La continuidad de esta función se justifica rápidamente por composición de funciones continuas (elevar a la cuarta es continua en todo $\mathbb{R}$, la raiz quinta no da problemas en $\mathbb{R}^{+} \cup \{0\}$ y el producto por escalares tampoco).

La derivabilidad tiene algo más de gracia. $g'(x) = \frac{4}{\sqrt[5]{x}}$ que claramente no está definida en el cero, donde cuenta con una asíntota.

Visto esto último, la función claramente no puede ser lipschitziana en ese intervalo porque, apoyándonos en la definición de derivada y en que esta no está acotada para $g(x)$, tenemos que $\forall \varepsilon > 0$, $\exists \delta > 0$ $/$ $\forall x_0 \in I$, $|x_0 - x| < 0 \Rightarrow |\frac{g(x_0)-g(x)}{x_0 - x} - g'(x)| < \varepsilon$ donde $x$ lo cogemos distinto de $0$ pero lo suficientemente cercano a este valor para que $|g'(x)| > K + \varepsilon$, donde $K$ sería una supuesta constante de Lipschitz que fracasa para cualquier valor positivo.
}
\item{
\textbf{Apartado 2} Justifica que si $D_1 = \{(t,x): x > 0\}$ y $D_2 = \{(t,x): x < 0\}$, para cada $(t_0, x_0) \in D_j$ y $j = 1, 2$, el (PVI) dado por $x' = g(x)$, $x(t_0) = x_0$ tiene solución maximal $x(t), t \in (\alpha, \omega)$ única con $(t, x(t))$, $t \in (\alpha, \omega)$ en el correspondiente dominio $D_j$. Calcula la solución maximal para $t_0 = 1$ y $x_0 = 32$, indicando cuál es su dominio.

En primer lugar, utilizando el Teorema de existencia y unicidad, tenemos garantizada la existencia y unicidad (¿qué si no?) de soluciones para cada punto de los $D_j$, $j = 1, 2$. Ahora solo tenemos que justificar que dichas soluciones pueden ser maximales.

Utilizando un corolario

}
\end{itemize}
\end{document}
